\section{Perturbation Theory}
We've been discussing scalar field theory for a while now, and this is because we want to use it as an example for how to use perturbation theory to study systems that are not analytically solvable.

We wish to calculate $\bra{\O}T\phi(x_1)\ldots\phi(x_n)\ket{\O}$ as a series in $\lambda$. These correlation functions tell us a lot about the theory. For example, the spectral theorem (which is a technical point which we may come back to later) tells us that we can use the two-point functions to deduce something about the energy and momentum of the field theory. From other correlation functions beyond two-point functions, we can use them to calculate scattering matrix element. Scattering ($S$) matrix elements describe a scattering event, where fields interact (go in and come and out in a reorganized form). The matrix elements give quantum amplitudes for reorganization - somewhat analogous to transition amplitudes that you would have computed in your quantum mechanics course. 

We have already started this process - we have already found the $\lambda = 0$ contribution, in the form of Wick's theorem. As a reminder, Wick's theorem told us that for the free field theory:
\begin{equation}
   \bra{\O}T\phi(x_1)\ldots\phi(x_n)\ket{\O} = \sum_{\text{pairings}}\prod_{\text{pairs}}\Delta(x_i, x_j)
\end{equation}
So for the interacting field theory:
\begin{equation}
    \bra{\O}T\phi(x_1)\ldots\phi(x_n)\ket{\O} = \sum_{\text{pairings}}\prod_{\text{pairs}}\Delta(x_i, x_j) + O(\lambda)
 \end{equation}
and we now want to probe the $O(\lambda)$ part. This is in some sense where the interesting things occur - if we neglect it, then the particles do not interact so the scattering matrix is highly uninteresting indeed (but of course this makes sense, as the $\lambda = 0$ limit is the free/non-interacting QFT). 

\subsection{Counterterms}
\emph{Lecture got cut short here due to connectivity issues - will continue on Monday}

