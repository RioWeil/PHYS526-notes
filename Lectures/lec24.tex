\section{Correlation Functions and the Generating Functional}

\subsection{Reviewing the Free Scalar Field Theory}

Last time, we learned how to do functional derivatives and functional integrals. In this lecture we study how to analyze a quantum field theory using functional methods. We will take an example of the simplest possible quantum field theory, that is of the real scalar field $\phi(x)$. 

We start with the free field theory - we will discuss interacting fields later. We have the action given by:
\begin{equation}
    S[\varphi] = \int d^4x \L(x)
\end{equation}
and the Lagrangian:
\begin{equation}
    \L(x) = -\frac{1}{2}\p_\mu \phi(x) \p^\mu\phi(x) - \frac{m}{2}\phi^2(x)
\end{equation}

Equivalently, we can describe the theory with the field equation (the Klein-Gordon equation):
\begin{equation}
    (-\p^2 + m^2)\phi(x) = 0.
\end{equation}
and the equal time-commutation relations of:
\begin{equation}
    [\phi(x), \dpd{}{y^0}\phi(y)]\delta(x^0 - y^0) = i\delta^4(x - y)
\end{equation}

We can solve this via the superposition of plane waves:
\begin{equation}
    \phi(x) = \int \frac{d^3k}{\sqrt{(2\pi)2k^0}}\left(e^{ik_\mu x^\mu}a(\v{k}) + e^{-ikx}a^\dag(\v{k})\right)
\end{equation}
and for $\phi$ to have the correct commutation relations, $a, a^\dag$ satisfy:
\begin{equation}
    [a(\v{k}), a^\dag(\v{l})] = \delta^3(\v{k} - \v{l})
\end{equation}
and note that $k^0 = \sqrt{\v{k}^2 + m^2}$. We can then define our basis of states starting with the vacuum state which satisfies:
\begin{equation}
    \braket{\O}{\O} = 1, a(\v{k})\ket{\O} = 0, \bra{\O}a^\dag(\v{k}) = 0
\end{equation}
and (in the case of bosons) have multi-particle states:
\begin{equation}
    \ket{\v{k}_1, \ldots \v{k}_n} = \frac{1}{\sqrt{n!}}a^\dag(\v{k}_1)\ldots a^\dag(\v{k}_n)\ket{\O}
\end{equation}
where the $\v{k}_1 \ldots \v{k}_n$ can be interchanged with each other as the $a^\dag$s commute. The energy is just the sum of the energies of each particle:
\begin{equation}
    P^0 = \sum_{j=1}^n \sqrt{\v{k}_j^2 + m}
\end{equation}
and the momentum the sum of the momenta:
\begin{equation}
    \v{P} = \sum_{j=1}^n \v{P}_n
\end{equation}

This is free-field theory; it is nice and solveable. We want to now cast it into a new form (we have discussed two in detail already, that being the field equation + commutation relations, and the Lagrangian). Namely, correlation functions!

\subsection{Correlation Function Recasting}
We consider $n$-point functions:
\begin{equation}
    \bra{\O}T\phi(x_1) \ldots \phi(x_n)\ket{\O}
\end{equation}
where $T$ is the time-ordering operator:
\begin{equation}
    T\phi(x_1)\phi(x_2) = \theta(x_1^0 - x_2^0)\phi(x_1)\phi(x_2) + \theta(x_2^0 - x_1^0)\phi(x_2)\phi(x_1)
\end{equation}
where $\theta$ is the Heaviside step function:
\begin{equation}
    \theta(x) = \begin{cases}
        1 & x > 0
        \\ 0 & x \leq 0
    \end{cases}
\end{equation}
Note that $T\phi(x_1)\phi(x_2)$ is symmetric in $x_1 \leftrightarrow x_2$! We can see this immediately in the 2-point function here, and it is easy to see that it generalizes to $n$-point functions, as the time-ordering will always reorganize the order of the $\phi$s regardless of how we might write the $x_i$s. 

It seems a bit strange to go from a compact definition of the field theory with the Lagrangian to an infinite number of correlation functions. It indeed would be totally silly if there was not a way to compactly write down the set of correlation functions. To this end, we will use functional techniques.

\subsection{The Generating Functional}
To begin, we search for a generating functional $Z[J]$. The way it is defined is if we take our $n$-point function, and multiply them by these functions $J$, integrate over the points, then sum to infinity and define the leading term as one. I.e.:
\begin{equation}
    Z[J] = 1 + \sum_{n=1}^\infty \frac{i^n}{n!}\int d^4x_1 \ldots d^4x_n J(x_1)\ldots J(x_n)\bra{\O}T\phi(x_1) \ldots \phi(x_n)\ket{\O}
\end{equation}
We can see that this contains all of our correlation functions. In order to get them out, we take functional derivatives and take $J$ to zero:
\begin{equation}
    \bra{\O}T\phi(x_1) \ldots \phi(x_n)\ket{\O} = \left.\frac{1}{i}\frac{\delta}{\delta J(x_1)} \ldots \frac{1}{i}\frac{\delta}{\delta J(x_n)}Z[J]\right|_{J=0}
\end{equation}
So if we can study this nice compact form. Note that there is an equivalent way of writing this as:
\begin{equation}
    Z[J] = 1 + \sum_{n=1}^\infty \frac{1}{n}\bra{\O}T(i\int d^4x J(x)\phi(x))^n \ket{\O}
\end{equation}
where we have correctly assumed the commutativity of the time-ordering and the integral. We can then use the definition of the exponential function and write:
\begin{equation}
    Z[J] = \bra{\O}Je^{i\int d^4x J(x)\phi(x)}\ket{\O}.
\end{equation}
We want to understand $Z[J]$ a little bit better.

\subsection{Finding a Functional Differential Equation}
In order to do so, we look for a functional differential equation it solves, and then solve that differential equation. To begin, we look at the first derivative of $Z[J]$:
\begin{equation}
    \frac{1}{i}\frac{\delta Z[J]}{\delta J} = \sum_{n=0}^\infty \frac{i^n}{n!}\int dx^1 \ldots dx_n J(x_1) \ldots J(x_n) \bra{\O}T\phi(x)\phi(x_1) \ldots \phi(x_n)\ket{\O}
\end{equation}

What I do next is to operate the wave operator (Klein-Gordon operator) on the above object. We are cavalier, and move the operator through all of the integrals. We might think that this results in the RHS vanishing, as $(-\p^2 + m^2)\phi = 0$. But this is not quite correct. This is because we have a time-derivative in the wave operator, and the time-ordering operator makes the product of fields have a discontinuity in the time. Thus we should get delta functions in time when we look at the time derivatives. Let's work it out:
\begin{equation}
    \begin{split}
        &(-\p^2 + m^2)\frac{1}{i}\frac{\delta Z[J]}{\delta J(x)} 
        \\ &= \sum_{n=0}^\infty \frac{i^n}{n!} \int dx_1 \ldots dx_n J(x_1)\ldots J(x_n) (-\p^2 + m^2)\bra{\O}T\phi(x)\phi(x_1) \ldots \phi(x_n)\ket{\O}
        \\ &= \sum_{n=0}^\infty \frac{i^n}{n!}\int dx_1 \ldots dx_n J(x_1) \ldots J(x_n)\left(\left(\dpd{}{x^0}\right)^2\bra{\O}T\phi(x)\phi(x_1) \ldots \phi(x_n)\ket{\O} - \bra{\O}T\left(\dpd{}{x^0}\right)^2 \phi(x)\phi(x_1) \ldots \phi(x_n)\ket{\O}\right)
        \\ &= \sum_{n=0}^\infty \frac{i^n}{n!} \int dx_1 \ldots dx_n J(x_1)\ldots J(x_n)\dpd{}{x^0}\bra{\O}\delta(x^0 - x_1^0)[\phi(x), \phi(x_1)] \ldots \ket{\O} + \ldots 
        \\ &= \sum_{n=0}^\infty \frac{i^n}{n!} \int dx_1 \ldots dx_n J(x_1)\ldots J(x_n)\sum_{i=1}^m -i\delta^4(x - x_i) \bra{\O}T\phi(x_1) \ldots \phi(x_{i-1})\phi(x_{i+1}) \ldots \phi(x_n)\ket{\O}
    \end{split}
\end{equation}
where we note that how the discontinuity due to the time ordering results in step functions, which when we take the time derivatives results in a dirac delta. Note the first time derivative commutes with the ordering, but the second time derivative gives us nonzero terms. The final expression has a sum over all of the $\phi$s, with the $\phi(x)$ dropping out and the $\phi(x_i)$ dropping out to be a dirac delta in each term. Using this, we can compare this back to the first term. One delta function per term liberates a $J$, and the rest sums up to the generating functional again. Hence, the slightly messy but absolutely correct discussion can be summarized in the result:
\begin{equation}
    (-\p^2 + m^2)\frac{1}{i}\frac{\delta}{\delta J(x)}Z[J] = J(x)Z[J]
\end{equation}
which is a functional differential equation for $Z[J]$. Now, if we use:
\begin{equation}
    \frac{1}{Z[J]}\frac{\delta}{\delta J(x)}Z[J] = \frac{\delta}{\delta J(x)}\ln(Z[J])
\end{equation}
so dividing the functional differential equation by $Z[J]$ we have:
\begin{equation}
    (-\p^2 + m^2)\frac{1}{i}\frac{\delta}{\delta J(x)}\ln Z[J] = J(x)
\end{equation}

\subsection{Solving the Functional Differential Equation}
We need to solve this equation; we do this by finding a Green function. It satisfies the equation:
\begin{equation}
    (-\p^2 + m^2)g(x, y) = \delta(x - y)
\end{equation}
They are notoriously ambiguous; for example we can add any solution the KG equation and we would still have a solution. Let assume we have fixed the ambiguity somehow, and have found a Green function. Then we can turn the derivative equation into an integral equation:
\begin{equation}
    \frac{\delta}{\delta J(x)}\ln Z[J] = \int i g(x, y)J(y) d^4y
\end{equation}
Note that an integrability condition of the above is that $g(x, y)$ is symmetric, else the RHS would not be a derivative of anything. 
Now, we can guess the functional anti-derivative of $\ln Z[J]$ to be:
\begin{equation}
    \ln Z[J] = \frac{i}{2}\int dxdy J(x)g(x, y)J(y) + C
\end{equation}
We can fix the constant by the boundary condition $Z[0] = 1$, so $C = 0$. Then, we can take the exponential of both sides to find:
\begin{equation}
    Z[J] = e^{\frac{i}{2}\int d^4x d^4y J(x)g(x, y)J(y)}
\end{equation}
so if we know the Green function, we have obtained a beautifully compact representation of the correlation functions! So, let's find it. What we do know is that two functional derivatives give us the two-point function:
\begin{equation}
    \bra{\O}T\phi(x_1)\phi(x_2)\ket{\O} = \left.-\frac{\delta^2}{\delta J(x_1)J(x_2)} Z[J]\right|_{J=0} = -ig(x_1, x_2)
\end{equation}
i.e. the two-point functions are Green functions! From this we obtain the specific Green function that we want; it is specifically the two-point function! To confirm this, let's operate the wave operator on it:
\begin{equation}
    (-\p^2 + m^2)\bra{\O}T\phi(x_1)\phi(x_2)\ket{\O} = \Delta(x_1, x_2) = -ig(x, y)
\end{equation}
We can use the time derivative and the product rule (keeping in mind the stepside functions that come from the time ordering) to find that indeed:
\begin{equation}
    (-\p^2 + m^2)\bra{\O}T\phi(x_1)\phi(x_2)\ket{\O} = -i\delta(x_1 - x_2)
\end{equation}
so this is a Green's function.
This is known as the \emph{Feynman propogator}. Note that in specifying it as the two-point function we have fixed a boundary condition that resolves the ambiguity in the Green function. We can get an explicit form of $\Delta(x_1, x_2)$ (down to some integrals which can't be evaluated fully - they evaluate to some modified Bessel functions) by plugging in the solution to the field theory:
\begin{equation}\label{eq-Delta}
    \Delta(x_1, x_2) = \Theta(x_1^0 - x_2^0) \int \frac{d^3k}{(2\pi)^3 2k^0}e^{ik(x_1 - x_2)} + \Theta(x_2^0 - x_1^0)\int \frac{d^3k}{(2\pi)^32k^0} e^{-ik(x_1 - x_2)}
\end{equation}
Then:
\begin{equation}
    -ig(x_1, x_2) = -i\int \frac{d^4k}{(2\pi)^4} \frac{e^{ik_\mu (x_1 - x_2)^\mu}}{k_\nu k^\nu + m^2 - i\e}
\end{equation}
Operating $-\p^2 + m^2$ on both sides, we can see that we get out $-\delta(x_1 - x_2)$ as required. The boundary condition is fixed by the $\e$ we put in the denominator. Specifically, note the sign here. Its role is to specify what happens when $k_\nu k^\nu = -m^2$. The $\e$ moves the real zeroes of $k$ to those with an imaginary part. 

Now, if we do the $k^0$ integral using Cauchy's integral formula (noting the poles in the complex plane off the real axis), one can show that the integral reproduces the above solution in Eq. \eqref{eq-Delta}. We don't go through it here; it is a technical exercise in complex analysis. 

When we find this, we are convinced that we know what $\Delta$ is. Given this concrete expression, and given that it is proportional to the Green function, we can come back to the generating functional and put in $\Delta$:
\begin{equation}
    \Delta(x, y) = \bra{\O}T\phi(x)\phi(y)\ket{\O} = -i \int \frac{d^4k}{(2\pi)^4}e^{ik(x - y)}\frac{1}{k^2 + m^2 - i\e}
\end{equation}
and this compact expression contains all information about the time-ordered correlation functions in our QFT! Just take a bunch of functional derivatives of it.

Note that in the free field theory case we could solve for it exactly, and found it to be a simple form - however it will still be a useful construct for when we discuss interacting fields, starting next lecture.
