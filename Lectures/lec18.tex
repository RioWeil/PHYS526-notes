\section{Dirac Field Theory II}
\subsection{The Dirac Equation}
We have been considering the Dirac equation:
\begin{equation}
    i\dpd{}{t}\psi = \left(i\gv{\alpha}\cdot\nabla + \beta m\right)\psi
\end{equation}
where $\alpha^a, \alpha^b, \alpha^c, \beta$ are Hermitian matrices (else the above operator would not be Hermitian) that obey an algebra. 

Why is this the right thing to do? For one, we can see that the above gives the correct dispersion relation (and the cool part - we can do this without even knowing what the matrices are! We only need to know the algebra of commutation relations we obey). How? We consider the square of the above operator, and use that the various terms anticommute. The square of the Hamiltonian has eigenvalues $p^2 + m^2$, so the Hamiltonian itself must have $\pm \set{p^2 + m^2}$. Alright, then how many eigenvalues are there? We can observe that $\alpha^1\alpha^2\alpha^3$ anticommutes with everything, and so when we operate them on the eigenvector of the Hamiltonian, we get another eigenvector but with the opposite sign. Explicitly:
\begin{equation}
    H\psi_\omega = \omega\psi_\omega
\end{equation}
where:
\begin{equation}
    \omega^2 = -\nabla^2 + m^2
\end{equation}
and so:
\begin{equation}
    H\alpha^1\alpha^2\alpha^3\psi_\omega = -\omega\alpha^1\alpha^2\alpha^3\psi_\omega
\end{equation}
In addition, $\alpha^1\alpha^2\alpha^3$ is nonsingular as $(\alpha^1\alpha^2\alpha^3)^{-1} = (\alpha^1\alpha^2\alpha^3)^\dag$. In fact we find that $\alpha^1\alpha^2\alpha^3$ has an equal number (2) of eigenvalues $+1$ and $-1$. So from this we find that there are an equal number of $+\sqrt{p^2 + m^2}$ and $-\sqrt{p^2 + m^2}$ eigenvalues for the Hamiltonian. This discussion is not a big deal, but is meant to show that playing with the algebra is quite profitable - this tells us that any representation of the algebra works. Someday, we may choose an explicit representation to solve the Dirac Equation, but for now let us keep things representation-free and consider space-time symmetry.

\subsection{Space-Time symmetry for the Dirac Field Theory}
Multiplying both sides of the Dirac equation by $i\beta$, we find:
\begin{equation}
    i\left(i\beta\dpd{}{t}\right)\psi = \left(i(i\beta \gv{\alpha})\cdot \nabla + im\right)\psi
\end{equation}
this looks more Lorentz covariant than it did in the previous formula. To emphasize this even further, let us rename $i\beta = \gamma^0$ and $i\beta \alpha^a = \gamma^a$. Since $\beta^2 = 1$ we find that $(\gamma^0)^2 = -1$. We can then rewrite the above equation as:
\begin{equation}\label{eq-diraceq}
    \left(\gamma^\mu \p_\mu + m\right)\psi = 0.
\end{equation}
where:
\begin{equation}
    \set{\gamma^\mu, \gamma^\nu} = 2\eta^{\mu\nu}.
\end{equation}
So the $\gamma$s here compactify the Dirac equation, and their commutation algebra in some sense encodes the algebra of the dirac matrices. We can even more consisely write the above using the Feynman slash notation $\dirac = \gamma^\mu \p_\mu$ to find:
\begin{equation}
    (\dirac + m)\psi = 0.
\end{equation}

So far we have just played with notation; let us now study the symmetries. Recall that the symmetries of the fields we have discussed (scalar, vector, tensor) have symmetries defined by their coordinate transformation properties. So we should do this for the Dirac field. But this is something we have not done yet; spinors don't have a representation as a field. It turns out that the Dirac field transforms like a scalar field - more technically it behaves like a spinor under a sort of ``changing of frames'' (informally phrased here...) but under transformations behaves like a scalar. It it interesting that somehow nature does not take on the simplest possible geometrical features. Already with the very common field describing (e.g.) electrons, the discussion has already become a bit complicated.

So, we have the Dirac field as in Eq. \eqref{eq-diraceq}, and we can study the space-time symmetries. If we just had the field equation, we would consider some infinitesimal transformation of the fields:
\begin{equation}
    \psi \to \psi + \delta \psi
\end{equation}
and this would be a symmetry if:
\begin{equation}
    (\dirac + m)\psi = 0 \implies (\dirac + m)\delta \psi = 0.
\end{equation}
Let's start with translation invariance.

\subsubsection{Translation Invariance}
Translation Invariance would correspond to:
\begin{equation}
    \delta_\mu \psi(x) = -\p_\mu \psi(x).
\end{equation}
Translation of course being generated by a derivative. It is easy to show that this satisfies the symmetry criterion:
\begin{equation}
    0 = \p_\mu \left(0\right) = \p_\mu \left((\dirac + m)\psi\right)\implies (\dirac + m)(\p_\mu \psi) = 0
\end{equation}
So this is shown to be a symmetry. Of course this is not the most useful way, as to apply Noether's theorem to determine the conserved current we require a Lagrangian for the theory. 

\subsubsection{Lorentz Invariance}
A lorentz transformation would correspond to:
\begin{equation}
    \delta \psi = \omega_{\mu\nu}\left(x^\mu \p^\nu - x^\nu \p^\mu\right)\psi + \mathbb{S} \psi
\end{equation}
where we need a matrix $\mathbb{S}$ in order to have the correct transformation. We then have (using the product rule - the $\dirac$ has derivatives!):
\begin{equation}
    (\dirac + m)\delta \psi = \omega_{\mu\nu}(\gamma^\mu \p^\nu - \gamma^\nu \p^\mu)\psi + \omega_{\mu\nu}\left(x^\mu \p^\nu - x^\nu \p^\mu\right)(\dirac + m)\psi + [\gamma^\mu, \mathbb{S}]\p_\mu \psi + \mathbb{S}(\dirac + m)\psi
\end{equation}
From the dirac equation the second term and fourth term vanish, so:
\begin{equation}
    (\dirac + m)\delta \psi = \omega_{\mu\nu}(\gamma^\mu \p^\nu - \gamma^\nu \p^\mu)\psi + [\gamma^\mu, \mathbb{S}]\p_\mu \psi
\end{equation}
The guess for $\mathbb{S}$ is:
\begin{equation}
    \mathbb{S} = -\omega_{\mu\nu}\frac{1}{4}[\gamma^\mu, \gamma^\nu]
\end{equation}
we could get this by process of elimination, considering all of the possible Hermitian matrices in our basis. Let's look at the commutator of $\gamma^\mu$ with $\mathbb{S}$:
\begin{equation}
    \begin{split}
        [\gamma^\mu, \mathbb{S}] &= -\frac{1}{4}\omega_{\rho\sigma}[\gamma^\mu, [\gamma^\rho, \gamma^\sigma]]
        \\ &= -\frac{1}{4}\omega_{\rho\sigma}\left(\gamma^\mu\gamma^\rho\gamma^\sigma - \gamma^\mu \gamma^\sigma \gamma^\rho - \gamma^\rho \gamma^\sigma \gamma^\mu + \gamma^\sigma \gamma^\rho \gamma^\mu\right)
        \\ &= -\frac{1}{4}\omega_{\rho\sigma}\left(\gamma^\mu\gamma^\rho\gamma^\sigma - \gamma^\mu \gamma^\sigma \gamma^\rho - \gamma^\rho \left(2\eta^{\sigma\mu} - \gamma^\mu \gamma^\sigma\right) + \gamma^\sigma \left(2\eta^{\rho\mu} - \gamma^\mu \gamma^\rho\right)\right)
        \\ &= -\frac{1}{4}\omega_{\rho\sigma}\left(\gamma^\mu\gamma^\rho\gamma^\sigma - \gamma^\mu \gamma^\sigma \gamma^\rho + \gamma^\rho \gamma^\mu \gamma^\sigma - \gamma^\rho \gamma^\mu \gamma^\rho - 2\gamma^\sigma \eta^{\sigma\mu} + 2\gamma^\sigma\eta^{\rho\mu}\right)
        \\ &= -\frac{1}{4}\omega_{\rho\sigma}(4)\left(\eta^{\mu\rho}\gamma^\sigma - \eta^{\mu\sigma}\gamma^\rho\right)
        \\ &= \omega_{\rho\sigma}(\gamma^\sigma\p^\rho - \gamma^\rho \p^\sigma)\psi
    \end{split}
\end{equation}
So we conclude:
\begin{equation}
    [\gamma^\mu, \mathbb{S}]\psi = -\omega_{\mu\nu}(\gamma^\mu \p^\nu - \gamma^\nu \p^\mu)\psi
\end{equation}
and therefore:
\begin{equation}
    (\dirac + m)\delta \psi = 0.
\end{equation}
So we have identified a symmetry!
\begin{equation}
    \delta \psi = \left(\omega_{\mu\nu}\left(x^\mu \p^\nu - x^\nu \p^\mu\right)\psi - \omega_{\mu\nu}\frac{1}{4}[\gamma^\mu, \gamma^\nu]\right)\psi
\end{equation}
However, again we don't have a systematic way of identifying the conserved Noether current. We need a Lagrangian density for that. Note that the space and time overlaps in the above (e.g. $[\gamma^0, \gamma^a]$) are Hermitian rather than anti-Hermitian. The boosts are not unitary. And there is a good reason for this; if they were, $\psi^\dag \psi$ would be invariant. But this is density, which is not a Lorentz scalar (it is rather the time component of a current). Lorentz symmetries are not like the usual non-relativistic symmetries where $\psi^\dag \psi$ is invariant, where a symmetry should not change the normalization of the wavefunction.

\subsection{Lagrangian Density for the Noether current}
So - we could try to find a Noether current. In order to do this, let's be pragmatic and find the Lagrangian density for the Dirac field theory (i.e. a Lagrangian density whose variation gives the Dirac equation).

We have the Dirac equation:
\begin{equation}
    (\gamma^\mu \p_\mu + m)\psi(x) = 0.
\end{equation}
so our Lagrangian density should look like:
\begin{equation}
    \begin{split}
        \L &= -i\psi^\dag \gamma^0\gamma^0 \dpd{}{t}\psi - \ldots
        \\ &= -i(\psi^\dag \gamma^0)(\dirac + m)\psi
        \\ &= -i\bar{\psi}(\dirac + m)\psi 
    \end{split}
\end{equation}
where we have defined $\bar{\psi} = \psi^\dag \gamma^0$. We strongly suspect this is Lorentz invariant (at least certainly its critical points are)... Applying Noether's theorem for translations is basically trivial:
\begin{equation}
    \delta\psi = -\p_\mu \psi \implies \delta \L = -\p_\mu \L
\end{equation}
and we can use this to find the conserved Noether's current. The Lorentz transformations takes a little more work, but not too much from our work above:
\begin{equation}
    \begin{split}
        &\delta \psi = \left(\omega_{\mu\nu}\left(x^\mu \p^\nu - x^\nu \p^\mu\right)\psi - \omega_{\mu\nu}\frac{1}{4}[\gamma^\mu, \gamma^\nu]\right)\psi
        \\ \implies & \delta \L = \omega_{\mu\nu}(x^\mu\p^\nu - x^\nu \p^\mu)\L = \p^\mu (-2\omega_{\mu\nu}x^\nu \L)
    \end{split}
\end{equation}
Our question for next time; are we able to write down a stress tensor that we could use for all of these symmetries?